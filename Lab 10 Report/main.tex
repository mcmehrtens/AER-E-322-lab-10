% AER E 361 Mission Report Template
% Spring 2023
% Template created by Yiqi Liang and Professor Matthew Nelson

% Document Configuration DO NOT CHANGE
\documentclass[12 pt]{report}
% --------------------LaTeX Packages---------------------------------
% The following are packages that are used in this report.
% DO NOT CHANGE ANY OF THE FOLLOWING OR YOUR REPORT WILL NOT COMPILE
% -------------------------------------------------------------------

\usepackage{hyperref}
\usepackage{parskip}
\usepackage{titlesec}
\usepackage{titling}
\usepackage{graphicx}
\usepackage{graphviz}
\usepackage[T1]{fontenc}
\usepackage{titlesec, blindtext, color} %for LessIsMore style
\usepackage{tcolorbox} %for references box
\usepackage[hmargin=1in,vmargin=1in]{geometry} % use 1 inch margins
\usepackage{float}
\usepackage{tikz}
\usepackage{svg} % Allows for SVG Vector graphics
\usepackage{textcomp, gensymb} %for degree symbol
\hypersetup{
	colorlinks=true,
	linkcolor=blue,
	urlcolor=cyan,
}
\usepackage{biblatex}
\addbibresource{main.bib}
\usepackage{amsmath}
\usepackage{listings}
\usepackage{multicol}
\usepackage{array}

\usepackage{hologo} %KYR: for \BibTeX
%\usepackage{algpseudocode}
%\usepackage{algorithm}
% This configures items for code listings in the document
\usepackage{xcolor}

\usepackage{fancyhdr} % Headers/Footers
\usepackage{siunitx} % SI units
\usepackage{csquotes} % Display Quote
\usepackage{microtype} % Better line breaks

\definecolor{commentsColor}{rgb}{0.497495, 0.497587, 0.497464}
\definecolor{keywordsColor}{rgb}{0.000000, 0.000000, 0.635294}
\definecolor{stringColor}{rgb}{0.558215, 0.000000, 0.135316}
\definecolor{mygreen}{rgb}{0,0.6,0}
\definecolor{mygray}{rgb}{0.5,0.5,0.5}
\definecolor{mymauve}{rgb}{0.58,0,0.82}

\lstdefinestyle{customc}{
  belowcaptionskip=1\baselineskip,
  breaklines=true,
  frame=L,
  xleftmargin=\parindent,
  language=C,
  showstringspaces=false,
  basicstyle=\footnotesize\ttfamily,
  keywordstyle=\bfseries\color{green!40!black},
  commentstyle=\itshape\color{purple!40!black},
  identifierstyle=\color{blue},
  stringstyle=\color{orange},
 }

 \lstset{ %
  backgroundcolor=\color{white},   % choose the background color; you must add \usepackage{color} or \usepackage{xcolor}
  basicstyle=\footnotesize,        % the size of the fonts that are used for the code
  breakatwhitespace=false,         % sets if automatic breaks should only happen at whitespace
  breaklines=true,                 % sets automatic line breaking
  captionpos=b,                    % sets the caption-position to bottom
  commentstyle=\color{commentsColor}\textit,    % comment style
  deletekeywords={...},            % if you want to delete keywords from the given language
  escapeinside={\%*}{*)},          % if you want to add LaTeX within your code
  extendedchars=true,              % lets you use non-ASCII characters; for 8-bits encodings only, does not work with UTF-8
  frame=tb,	                   	   % adds a frame around the code
  keepspaces=true,                 % keeps spaces in text, useful for keeping indentation of code (possibly needs columns=flexible)
  keywordstyle=\color{keywordsColor}\bfseries,       % keyword style
  language=Python,                 % the language of the code (can be overrided per snippet)
  otherkeywords={*,...},           % if you want to add more keywords to the set
  numbers=left,                    % where to put the line-numbers; possible values are (none, left, right)
  numbersep=8pt,                   % how far the line-numbers are from the code
  numberstyle=\tiny\color{commentsColor}, % the style that is used for the line-numbers
  rulecolor=\color{black},         % if not set, the frame-color may be changed on line-breaks within not-black text (e.g. comments (green here))
  showspaces=false,                % show spaces everywhere adding particular underscores; it overrides 'showstringspaces'
  showstringspaces=false,          % underline spaces within strings only
  showtabs=false,                  % show tabs within strings adding particular underscores
  stepnumber=1,                    % the step between two line-numbers. If it's 1, each line will be numbered
  stringstyle=\color{stringColor}, % string literal style
  tabsize=2,	                   % sets default tabsize to 2 spaces
  title=\lstname,                  % show the filename of files included with \lstinputlisting; also try caption instead of title
  columns=fixed                    % Using fixed column width (for e.g. nice alignment)
}

\lstdefinestyle{customasm}{
  belowcaptionskip=1\baselineskip,
  frame=L,
  xleftmargin=\parindent,
  language=[x86masm]Assembler,
  basicstyle=\footnotesize\ttfamily,
  commentstyle=\itshape\color{purple!40!black},
}

\lstset{escapechar=@,style=customc}

\titlelabel{\thetitle.\quad}

% From here on out you can start editing your document
\newcommand{\subtitle}[1]{%
  \posttitle{%
    \par\end{center}
    \begin{center}\LARGE#1\end{center}
    \vskip0.5em}%
}

\title{\textbf{Iowa State University
\\{\Large Aerospace Engineering}}}
\subtitle{AER E 322 Lab 10\\
		  Structure Model Building}
\author{Matthew Mehrtens, Peter Mikolitis, and Natsuki Oda}

\newcommand{\etal}{\textit{et al}., }
\newcommand{\ie}{\textit{i}.\textit{e}., }
\newcommand{\eg}{\textit{e}.\textit{g}., }

% Define the headers and footers
\setlength{\headheight}{70.63135pt}
\geometry{head=70.63135pt, includehead=true, includefoot=true}
\fancypagestyle{plain}{
	\fancyhead{}\fancyfoot{} % clears the headers/footers
	\fancyhead[L]{\textbf{AER E 322}}
	\fancyhead[C]{\textbf{Aerospace Structures Laboratory Report}\\
					 \textbf{Lab 10 Structure Model Building}\\
					 Section 4 Group 2\\
					 Matthew Mehrtens, Peter Mikolitis, and Natsuki Oda\\
					 \today}
	\fancyhead[R]{\textbf{Spring 2023}}
	\fancyfoot[C]{\thepage}
}
\pagestyle{fancy}
\fancyhead{}\fancyfoot{} % clears the headers/footers
\fancyhead[L]{\textbf{AER E 322}}
\fancyhead[C]{\textbf{Aerospace Structures Laboratory Report}\\
			  \textbf{Lab 10 Structure Model Building}\\
			  Section 4 Group 2\\
			  Matthew Mehrtens, Peter Mikolitis, and Natsuki Oda\\
			  \today}
\fancyhead[R]{\textbf{Spring 2023}}
\fancyfoot[C]{\thepage}

\begin{document}
\maketitle
\tableofcontents

\chapter{Pre-Lab} \label{pre-lab}
\section{Introduction} \label{introduction}
% TODO: Revise
Our group was tasked with designing and building an aircraft structure utilizing the PASCO tool kit/building kit. To do so, we utilized the ``rapid design, prototyping, and learning'' concept. This would allow our group to rapidly come up with a design, then revise the design to get to a structure that was suitable for collecting data. We decided on a fuselage that we made out of two octagons connected by long members. We chose to use long members for connecting the octagons, so the structure would be less stable, in theory providing better data. The octagon was chosen because it was close enough to a circle without making the circumference too large. Our group performed two tests on the fuselage, each with \num{5} rounds lasting \qty{15}{\second}. Each test had five rounds of testing where each run had an increase in frequency of the wave driver oscillation (\qtylist{1;2;5;10;15}{\hertz}). The first test was set up to apply a tension force to the system utilizing two wave drivers. The wave drivers were set to pull on the system simultaneously to provide the maximum tension force. The second test used only one wave driver pulling down on one side of the fuselage to simulate a torsion force. Three \qty{5}{\newton} load cells were attached to the members connecting each octagon together. This would allow our group to collect data on different members while performing each test to. Multiple load cells were used on different members to clearly observe the force distribution across the entire system. Two were placed on each side of the fuselage, and one on the bottom.

\section{Objectives} \label{objectives}
% TODO: Revise
\begin{itemize}
	\item Utilize the ``rapid design, prototyping, and learning'' concept to simulate a part of an aircraft structure.
	\item Create an aircraft structure that is flexible enough to collect usable data, \ie flexible enough to have a wave driver apply enough force for deflection.
	\item Determine if load cells can be used to recover frequency of oscillations.
	\item Observe how oscillatory forces are transferred throughout the system.
\end{itemize}

\section{Hypothesis} \label{hypothesis}
% TODO: Revise
\textbf{Test 1}

For test one, we expect to see similar data from all four load cells. Since the system homogeneous, we expect the load to be distributed evenly between the connection members no matter where the load is applied to the system. While the load is applied, the load cells should read out a tensile load due to the whole system being in tension. We also expect runs \numrange{1}{3} to provide the best data. This is because the load cells are only sampling at \qty{20}{\hertz}, so the wave driver oscillation frequency of tests \numlist{4;5} are too high for the load cells to accurately capture data.  

\textbf{Test 2}

Test two will provide different data across the load cells. For the load cell on the vertical part of the octagon on the member which the force is acting, we expect to see higher tensile loads because the system is clamped at the top as the wave driver is pulling down on directly on the load cell. For the other three load cells on, the outputs will still be tension, but the magnitude will be lesser than the load cell directly on the line of action. This is because we expect there to be more energy loss compared to test \num{1}. Similar to test \num{1}, runs \numrange{1}{3} of testing will provide the best data and runs \numlist{4;5} will provide the worst due to the same reason listed above.  

\chapter{Lab Work} \label{lab_work}
\section{Variables} \label{variables}
\subsection{Independent Variables} \label{variables-independent_variables}
\begin{itemize}
	\item Fuselage structure, \eg the length and radius of the fuselage
	\item Frequency of the wave driver
	\item Amplitude or voltage of the wave driver
	\item Location of wave driver(s)
	\item Sampling frequency
\end{itemize}

\subsection{Dependent Variables} \label{variables-dependent_variables}
\begin{itemize}
	\item Force measured by load cells
\end{itemize}

\section{Work Assignments} \label{work_assignments}
Refer to Table \ref{table:work_assignments} for the distribution of work during this lab.

\begin{table}[!htbp]
\caption{Work assignments for AER E 322 Lab 10.}
\begin{center}
	\begin{tabular}{|c|c|c|c|}
		\hline
		\multicolumn{1}{|c|}{\textbf{Task}}&\textbf{Matthew}&\textbf{Peter}&\textbf{Natsuki}\\
		\hline
		\multicolumn{4}{|c|}{\textit{Lab Work}}\\
		\hline
		Data Recording&X&X&X\\
		\hline
		Exp. Setup&X&X&X\\
		\hline
		Exp. Work&X&X&X\\
		\hline
		Exp. Clean-Up&X&X&X\\
		\hline
		\multicolumn{4}{|c|}{\textit{Post Lab}}\\
		\hline
		Data Analysis&X&&\\
		\hline
		\multicolumn{4}{|c|}{\textit{Report}}\\
		\hline
		Introduction&&X&\\
		\hline
		Objectives&&X&X\\
		\hline
		Hypothesis&X&X&\\
		\hline
		Variables&&&X\\
		\hline
		Materials&&X&X\\
		\hline
		Apparatus&&X&X\\
		\hline
		Procedures&X&X&\\
		\hline
		Data&X&&\\
		\hline
		Analysis&X&X&X\\
		\hline
		Conclusion&X&&\\
		\hline
		References&X&&\\
		\hline
		Appendix&X&&\\
		\hline
		Revisions&X&X&X\\
		\hline
		Editing&X&&\\
		\hline
	\end{tabular}
\end{center}
\label{table:work_assignments}
\end{table}

\section{Materials} \label{materials}
% TODO

\section{Apparatus} \label{apparatus}
% TODO

\section{Procedures} \label{procedures}
\subsection{Setup} \label{procedures-setup}
% TODO

\subsection{Cleanup} \label{procedures-cleanup}
% TODO

\section{Data} \label{data}
% TODO

\chapter{Conclusion} \label{conclusion-chapter}
\section{Analysis} \label{analysis}
% TODO

\section{Conclusion} \label{conclusion-section}
% TODO

\printbibliography[heading=subbibintoc]
\appendix
\end{document}
